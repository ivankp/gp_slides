\renewcommand{\vec}[1]{\mathbf{#1}}
\newcommand{\N}[1]{\mathcal{N}\left(#1\right)}
\newcommand*{\T}{^{\mkern-1.5mu\mathsf{T}}}
\newcommand{\ud}[1]{\underline{\smash{#1}}}

\frame{\frametitle{Introduction}
% \includegraphics[width=\textwidth]{pages/cur_pT_yy}
}

\frame{\frametitle{Definition}
\begin{itemize}
  \item One of the most commonly seen defenitions of a GP is this:
\end{itemize}
\begin{tcolorbox}
  A Gaussian process is a collection of random variables, any finite number
  of which have a joint Gaussian distribution.
\end{tcolorbox}
\begin{itemize}
  % I hate it, because it provides no intuition about the usefulness of the
  % mathematical object or the inference method that utilizes it.
  \item This is like saying that a fork is something that splits at one end. \\
  An accurate, yet uninspiring statement.

  \vspace{5mm}
  \item We'll return to the definition later.
\end{itemize}
}

\begin{frame} \frametitle{The Setup}
\begin{itemize}
  \item Assume we made an observation of a vector
    $\vec{d} = [y_1,y_2,\mathellipsis,y_n]$,\\
    and that $\vec{d}$ is sampled from a multivariate Gaussian
    distribution,\\i.e. $\vec{d}\sim\N{\vec{0},D}$.
  \vspace{1mm}
  \item Assuming non-zero mean is no more general, because the mean can be
    absorbed into the defenition of $\vec{d}$.
  \vspace{1mm}
  \item If we arbitrarily split $\vec{d}$ into subvectors
    $\vec{a}$ and $\vec{b}$, then we can write
\end{itemize}
\begin{equation}\label{eq:n1}
  \begin{bmatrix} \vec{a} \\ \vec{b} \end{bmatrix} \sim
    \N{\vec{0},\begin{bmatrix} A & C \\ C\T\!\! & B \end{bmatrix}}
\end{equation}
\begin{itemize}
  \item The conditional probability of $\vec{b}$ given $\vec{a}$ is
\end{itemize}
\vspace{1mm}
\begin{equation}
  p(\vec{b}|\vec{a})
  = \frac{p(\vec{a}\cap\vec{b})}{p(\vec{a})}
  = \frac{p(\vec{d})}{\int p(\vec{d})p(\vec{b}') d\vec{b}'}
  = \boxed{ \N{C\T\!A^{-1}\vec{a},\,B-C\T\!A^{-1}C} }
\end{equation}
\vspace{-5mm}
\begin{itemize}
  \item Proof: {\small\url{https://stats.stackexchange.com/q/30588/239215}}
\end{itemize}
\end{frame}

\begin{frame} \frametitle{Bayesian Inference}
\vspace{-2mm}
\begin{itemize}
  \item Recall the Bayes' theorem
\end{itemize}
\vspace{1mm}
\begin{equation}
  \text{posterior} =
  \frac{\text{likelihood}\times\text{prior}}{\text{marginal likelihood}},
  \quad
  p(\vec{b}|\vec{a}) = \frac{p(\vec{a}|\vec{b})}{p(\vec{a})}\,p(\vec{b}).
\end{equation}
\begin{itemize}
  \item $p(\vec{b}) = \N{\vec{0},B}$ can be viewed as the \ud{prior},
    and $p(\vec{b}|\vec{a}) = \N{C\T\!A^{-1}\vec{a},\,B-C\T\!A^{-1}C}$ as the
    \ud{posterior}. \\
    {\small\color{gray} * Conditioning}
  \vspace{1mm}
  \item GP is defined by the multivariate Gaussian distribution in
    Eq.~\ref{eq:n1}.
  \vspace{1mm}
  \item Clearly, we didn't have to have observed the $\vec{b}$ part of the
    vector to make this inference.
  \vspace{1mm}
  \item In GP regression, inference is made about unobserved function values.
  \begin{itemize}
    \item Think of a function as a (generally continuous infinite-dimensional)
      vector, whose elements are labeled by the coordinates on the manifold
      on which the function lives.
  \end{itemize}
  \vspace{1mm}
  \item Non-zero convariance, $C$, contains the additional information. \\
    Otherwise, $p(\vec{b}) = p(\vec{b}|\vec{a})$.
\end{itemize}
\end{frame}

\begin{frame} \frametitle{GP Regression}
\begin{itemize}
  \item Given: observed function values, $y_i$, at points $x_i$.
  \item Model: prior distribution of the function values,
    given my the mean $m$ and covariance matrix $K$.
    This is the GP.
\end{itemize}
\begin{equation}
  f(x) \sim \N{m(x),K(x,x')}
\end{equation}
  \vspace{-7mm}
\begin{itemize}
  \item Take $m = 0$ for brevity
  \item For observed values $y$, and unobserved values $y_*$, we can
    write
\end{itemize}
\begin{equation}
  \begin{bmatrix} y \\ y_* \end{bmatrix} \sim
  \N{0,\,\begin{bmatrix} K(x,x) & K(x,x_*) \\ K(x_*,x) & K(x_*,x_*) \end{bmatrix}},
  \ \text{or}
\end{equation}
\begin{equation}
  \begin{bmatrix} y \\ y_* \end{bmatrix} \sim
    \N{0,\,\begin{bmatrix} K & K_* \\ K_*\T\!\! & K_{**} \end{bmatrix}}.
\end{equation}
  \vspace{-4mm}
\begin{itemize}
  \item Then,
\end{itemize}
\begin{equation}
  \boxed{ y_*|y \sim \N{K_*\T K^{-1}y,\,K_{**}-K_*\T K^{-1}K_*} }
\end{equation}
\end{frame}

\begin{frame} \frametitle{GP Regression}
\begin{equation}
  \boxed{ y_*|y \sim \N{K_*\T K^{-1}y,\,K_{**}-K_*\T K^{-1}K_*} }
\end{equation}
\begin{itemize}
  \item Gives the best linear unbiased prediction.
  \item Direct measure of uncertainty at each point of the function.
  \item Function value predictions are independent.
  \begin{itemize}
    \item Can compute point-by-point. \hspace{5mm}
    {\small\color{gray} * Lazy learning}
  \end{itemize}
  \item Also called Kriging.
  \begin{itemize}
    \item The theoretical basis for the method was developed by the French
      mathematician Georges Matheron in 1960, based on the Master's thesis of
      Danie G. Krige, the pioneering plotter of distance-weighted average gold
      grades at the Witwatersrand reef complex in South Africa.
      [\href{https://en.wikipedia.org/wiki/Kriging}{Wikipedia}]
  \end{itemize}
\end{itemize}
\end{frame}

\begin{frame} \frametitle{GP Regression Algorithm}
\begin{equation}
  \boxed{ y_*|y \sim \N{K_*\T K^{-1}y,\,K_{**}-K_*\T K^{-1}K_*} }
\end{equation}
\begin{equation}
  \bar{y}_* = \vec{k}_*\T K^{-1} \vec{y}, \quad
  \mathrm{var}(y_*) = k_{**} - \vec{k}_*\T K^{-1} \vec{k}_*
\end{equation}
\vspace{-3mm}
\begin{itemize}
  \item In practice, instead of inverting the $K$ matrix, \\
    Cholesky decomposition is used: $K = LL\T$.
  \begin{itemize}
    \item $K$ is symmetric.
    \item $L$ is triangular.
  \end{itemize}
  \vspace{1mm}
  \item Then $K^{-1} = (L^{-1})\T L^{-1}$, so
\end{itemize}
\begin{equation}
  \bar{y}_* = \left(L^{-1}\vec{k}_*\right)\T \left(L^{-1} \vec{y}\right),
  \quad
  \mathrm{var}(y_*) = k_{**} -
    \left(L^{-1}\vec{k}_*\right)\T
    \left(L^{-1}\vec{k}_*\right)
\end{equation}
\vspace{-5mm}
\begin{itemize}
  \item Cholesky decomposition \plus\ back substitution generally has better
    numerical stability than matrix inversion \plus\ multiplication.
  \vspace{1mm}
  \item My \cpp\ implementation:
    % {\small\url{https://github.com/ivankp/GP/blob/master/include/gp.hh}}
    {\small\url{https://git.io/fhN1j}}
\end{itemize}
\end{frame}

\begin{frame} \frametitle{What is a process?}
\end{frame}

\begin{frame} \frametitle{GP: Function space view}
\end{frame}

\begin{frame} \frametitle{GP: Weight space view}
\begin{itemize}
  \item \red{kernel trick}
\end{itemize}
\end{frame}

\newcommand{\plot}[2][0.36]{%
  \includegraphics[width=#1\textwidth]{gp/{#2}.pdf}
}

\begin{frame} \frametitle{Squared Exponential Kernel}
\textbf{Synonims}: Radial Basis Function, Gaussian, Exponentiated Quadratic
\begin{equation}
  k(x,x') = \exp\left(-\frac{\left|x-x'\right|^2}{2\ell^2}\right)
\end{equation}
\begin{itemize}
  \item The de-facto default kernel.
  \item Produces infinitely differentiable functions.
  \item $\ell$ - lengthscale. Determines the scale of function's features.
    Generally, cannot extrapolate farther than distance $\ell$ from the data.
\end{itemize}
\begin{center}
  \plot[0.5]{js_se_0_1_1.5}
\end{center}
\end{frame}

\begin{frame} \frametitle{Noise and signal variances}
\vspace{-2mm}
\begin{itemize}
  \item Typically, two extra parameters are added to the kernel model.
  \item $\sigma_s^2$ - signal variance.
    This is a multiplicative factor attached to every additive term in a
    kernel.
    Determines the average distance of the function away from the mean.
  \item $\sigma_n^2$ - noise variance.
    This is added to the diagonal elements of the kernel, and provides a
    measure of the observation's uncertainty.
  \item With both terms introduced, the SE kernel looks like this
\end{itemize}
\begin{equation}
  k(x,x') = \sigma_s^2 \exp\left(-\frac{\left|x-x'\right|^2}{2\ell^2}\right)
          + \sigma_n^2 \delta_{xx'}
\end{equation}
\vspace{-2.5mm}
\begin{itemize}
  \item $\sigma_n^2$ can be different for every point, e.g. Poisson
    uncertainties for a histogram.
\end{itemize}
\vspace{-4mm}
\begin{changemargin}{-4.3mm}{0mm}
\plot{rw_se_0.1_1_1}
\plot{rw_se_5e-05_1.08_0.3}
\plot{rw_se_0.89_1.16_3}
\end{changemargin}
\end{frame}

\begin{frame} \frametitle{Kernel parameter optimization}
\end{frame}

\begin{frame} \frametitle{GP for Classification}
\end{frame}

\begin{frame} \frametitle{- - -}
\begin{itemize}
  \item GP regression is not very good for extrapolation.
  \item \red{connection to neural nets}
\end{itemize}
\end{frame}

\newcommand{\R}[2]{\item #1 \\ {\scriptsize\url{#2}}}
\newcommand{\arxiv}[1]{arXiv:\href{https://arxiv.org/abs/#1}{#1}}

\begin{frame} \frametitle{References}
\begin{itemize}
  \R{Rasmussen \& Williams (the cannonical textbook)}
    {http://www.gaussianprocess.org/gpml/chapters/}
  \R{Mark Ebden {\small(simpler than R\&W)}}
    {https://arxiv.org/abs/1505.02965}
  \R{Katherine Bailey {\small(simple python\,\plus\,numpy code)}}
    {http://katbailey.github.io/post/gaussian-processes-for-dummies/}
  \R{CS229 Stanford Notes (Chuong Do)}
    {http://cs229.stanford.edu/section/cs229-gaussian_processes.pdf}
  \R{Kernel Cookbook (David Duvenaud)}
    {https://www.cs.toronto.edu/~duvenaud/cookbook/}
  \R{Rob Fletcher's slides (ATLAS internal)}
    {https://indico.cern.ch/event/744257/contributions/3077617/attachments/1688169/2715519/GP_Fletcher.pdf}
  \item Some interesting papers:
  \begin{itemize}
    \item \arxiv{1709.05681} (Modeling smooth backgrounds)
    \item \arxiv{1302.4245} (Spectral mixture kernel)
  \end{itemize}
\end{itemize}
\end{frame}

